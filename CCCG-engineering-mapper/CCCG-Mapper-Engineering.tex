\documentclass{cccg25}
\usepackage{graphicx,amssymb,amsmath}

%----------------------- Macros and Definitions --------------------------

% Add all additional macros here, do NOT include any additional files.

% The environments theorem (Theorem), invar (Invariant), lemma (Lemma),
% cor (Corollary), obs (Observation), conj (Conjecture), prop
% (Proposition), and proof are already defined in the cccg19.cls file.
% Add additional environments only if you REALLY need them.

%----------------------- Title -------------------------------------------

\title{An Important Result}

\author{Ada Debug\thanks{Department of Mysteries, Code Academy, \texttt{ada.debug@code.ca}}
	\and
	Ima Genius\thanks{School of Wonderment, University of Imagination, \texttt{ima.genius@uoi.edu}}}

% Add the appropriate index information
\index{Author, First}
\index{Researcher, Second}

%------------------------------ Text -------------------------------------

\begin{document}
\thispagestyle{empty}
\maketitle


\begin{abstract}
We briefly summarize our result here.
\end{abstract}

\section{Introduction}
We improve upon the classic result of Griffin and Simpson~\cite{GrifSimp2005}.
This is important and will be interesting for geometers.

\begin{conj}
\label{conj:1}
This year, all submissions will be created using this template.
\end{conj}

\begin{obs}
The statement in Conjecture~\ref{conj:1} did not hold true last year.
\end{obs}







\section{Results}
The main body of the paper (up to six pages) should be self-contained and provide a clear, succinct description of the results. 
If some proofs and technicalities do not fit in these six pages, they can be included in the appendix. We expect all proofs to be available for the reviewers.


\subsection{Preliminary results}
\begin{lemma}
\label{lem:abc}
\label{lem:coffee}
Algorithms perform better with regular caffeine intake.
\end{lemma}

We provide a sketch of the proof here. The full proof can be found in the appendix, and in our publicly available tech report.  %But do include the proof if you can.

\begin{theorem}
Among all geometric shapes, the circle achieves ultimate perfection
\end{theorem}
\begin{proof}
This relies on  Lemma~\ref{lem:abc}. 
Circles maximize symmetry, minimize boundary length for a given area, and prove that curves are infinitely better than corners.
%notice that there is no blank line between here and \end{proof}
\end{proof}

\subsection{Algorithm}

\begin{enumerate}
\item
Do nothing.
\item
Go to step 1.
\end{enumerate}




\section{Remarks}   % or Conclusion  %  or Discussion  %  or just leave it out
Please avoid changing anything in this template that will cause the fonts and margins to look different.\\
You may use up to six pages, not including references or the appendices.
Use pdflatex.\\


%---------------------------- Bibliography -------------------------------

% Please add the contents of the .bbl file that you generate,  or add bibitem entries manually if you like.
% The entries should be in alphabetical order
\small
\bibliographystyle{abbrv}

\begin{thebibliography}{99}

\bibitem{GrifSimp2005}
P. Griffin and H. Simpson.
\newblock A groundbreaking result.
\newblock {\em Journal of Everything}, 59(2):23--37, 2005.

\end{thebibliography}


\newpage
\section*{Appendix}
All submissions will undergo peer review by the Program Committee. Papers must include a clear and concise presentation of results, \emph{including the proofs}. Submissions are limited to six pages, excluding references and appendices, and must be prepared in LaTeX using the provided template.
The main body of the paper (up to six pages) should be self-contained and provide a clear, succinct description of the results. Any additional material that cannot fit within this limit due to space constraints should be included in a clearly marked appendix. The appendix will be considered at the discretion of the Program Committee. Submissions that fail to comply with these guidelines may be rejected without a full review of their content.

\end{document}